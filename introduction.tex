\chapter*{序}
\markboth{\textsc{序}}{}
\addcontentsline{toc}{chapter}{序}
\setcounter{page}{1}
\pagenumbering{arabic}


\emph{同伦类型论} 是一个全新的数学分支, 其以出人意料的方式结合了多个不同的领域。它基于近期发现的\emph{同伦论}和\emph{类型论}之间的联系。
同伦论是代数拓扑与同调代数的产物, 与高阶范畴论有着联系; 而类型论则是数理逻辑和理论计算机科学的一个分支。
尽管二者之间的联系仍是研究的重点, 但愈发清晰的是, 这只是一个需要更多时间和努力来理解的学科的开端。
它还会涉及到像球面的同伦群, 类型检查算法, 弱$\infty$-群胚的定义这样看起来十分遥远的话题。

同伦类型论也为数学基础带来了新思想。
\index{foundations, univalent}%
一方面, 我们有Voevodsky提出的精美的\emph{泛等公理}
\index{univalence axiom}%
由泛等公理, 我们可以得到, 同构的结构实际上可以是恒等的, 这是数学家在工作时一直乐于使用的一个原则, 即使其与传统的``官方''的数学基础并不兼容。
另一方面, 我们有\emph{高阶归纳类型}, 其提供了对同伦论中基本空间和构造的直接、逻辑化的描述: 如球面, 柱体, 截断, 局域化, 等等。
这些思想在经典的集合论的基础中是不可能被直接捕捉到的。但当它们在同伦类型论中结合在一起时, 一种全新的``同伦类型的逻辑''浮现了出来。
\index{foundations}%

这表明了对数学基础的一种新的洞见, 其具有内在的, 直觉化的同伦论的内容——
即数学对象的``不变量''的概念——与便捷的机器实现。后者可以为数学家的工作提供实用的帮助。
这就是\emph{泛等基础}纲领。本书被设计为对泛等基础的基本概念进行表述的第一个系统性文献。
同时也是通过这种新方式进行推理的例子的合集——但是不需要读者了解或学习任何形式逻辑, 或是使用任何的证明助手。

% This enlarges the page by one line in letter format. Use sparringly.
\OPTwidow

需要强调的是, 同伦类型论还是一个新兴的领域, 而泛等基础也是一个还在进行中的工作。
这本书应被视作这个对这个领域的一部分, 在写作时拍摄的一张``快照'', 而不是经过润色的, 对一个完整学说进行的阐释。
就像我们稍后会大致讨论的, 同伦类型论的许多方面都还未被充分地理解——而还有一些甚至不会在此处被提及。
几乎可以肯定的是, 尽管最终的理论和本书所描述的理论不会完全相同, 但其 \emph{至少} 会和这个理论一样有力; 因此, 
我们相信泛等基础最终将成为集合论的一个切实可行的替代, 作为大多数数学家进行的非形式化数学的``隐形基础''。

\subsection*{类型论}

类型论最初由Bertrand Russell\cite{Russell:1908}, \index{Russell, Bertrand}发明, 作为一种遏制当时在数学基础研究中出现的悖论的手段。
在接下来的几十年里, 它被许多人进一步发展, 其中Church~\cite{Church:1940tu, Church:1941tc}将他的\textit{$\lambda$-演算}与类型论相结合。
尽管其还未被广泛视作经典数学的基础——集合论在此更加流行——类型论仍然由众多的应用, 特别是在计算机科学与编程语言理论~\cite{Pierce-TAPL}。

\index{programming}%
\index{type theory}%
\index{lambda-calculus@$\lambda$-calculus}%

Per Martin-L\"{o}f \cite{Martin-Lof-1972,Martin-Lof-1973,Martin-Lof-1979,martin-lof:bibliopolis}, 
与其他人一起发展了对Church的类型论修改后的一种``谓词性''版本。现在通常称其为一个依值的, 构造性的, 直觉主义的类型论, 
或简称为\emph{Martin\--L\"of类型论}。这是我们在本书中所考虑的系统的基础; 其一开始被视作对构造性数学进行形式化的严谨框架。
之后, 我们通常使用``类型论''来特别指代这一系统、或与其相似的系统, 尽管类型论这个学科涵盖的概念更加广泛。(见\cite{somma, kamar}以了解类型论的历史)

类型论与集合论不同的不同之处在于, 我们使用\emph{类型}这一原始概念来分类对象, 就像编程语言中所使用的数据类型一样。
这些精巧的结构化的类型被用于表示被分类对象的具体规范, 并提供对这些对象进行推理的规则。举一个简单的例子, 
众所周知, 具有积类型$A \times B$的对象的形式是$\pairr{a, b}$, 因此人们自然知道该如何构造和分解它们。类似地, 通过一个使用类型为$A$的对象编码的, 
具有类型$B$的对象, 我们可以得到具有函数类型$A \to B$的对象。并且给定一个类型为$A$的参数, 其值可以被计算。
这种所有对象均有的, 坚实可预测的行为(与集合论中更加自由的形成规则相对, 其允许非齐次集的存在), 使类型论在检验计算机程序正确性方面得以广泛应用。
而其清晰的推理规则, 而与类型的构造相结合, 形成了现代\emph{证明助手}的基础, 
\index{proof!assistant}%
\indexsee{computer proof assistant}{proof assistant}
\index{mathematics!formalized}%
其在形式化数学与检验形式化证明正确性中被应用。我们将在下文继续探讨类型论的这一方面。

然而, 从数学角度理解类型论始终存在一个难题, 类型论的基本概念\emph{类型}与\emph{集合}概念在本质上的差异一直难以被精确刻画。
我们相信, 将类型不再视为某种奇异集合(或许无需经典逻辑构造而成), 而是从同伦论视角将其理解为空间, 这一新思路是一次重大突破。
尤其值得关注的是, 它解决了长久以来的困惑: 为何类型中元素的相等性概念会与集合中元素的相等性存在根本差异。

在同伦论中, 人们关注的是空间
\index{topological!space}%
与它们之间在同伦意义上的连续映射。
\index{function!continuous!in classical homotopy theory}%
在一对连续映射 $f : X \to Y$与$g : X\to Y$之间, 如果一个连续映射$H : X \times [0,  1] \to Y$
满足$H(x, 0) = f (x)$和$H(x, 1) = g(x)$,我们就称其为\emph{同伦}。同伦$H$可以被视作一种从$f$到$g$的``连续形变''。
\index{homotopy!topological}%
如果存在一堆方向相反的连续映射, 其复合同伦于单位映射, 我们就说空间$X$和$Y$是\emph{同伦等价}的$\eqv X Y$,
\index{homotopy!equivalence!topological}%
即,它们在``同伦意义上''是同构的。同伦等价的空间具有相同的代数不变量(例如,同调群或基本群),因此称它们具有相同的\emph{同伦类型}。

\subsection*{同伦类型论}

同伦类型论(HoTT)从同伦论的视角解释类型论。
同伦类型论中, 我们将类型视作``空间''或(像同伦论已研究的那样)视作高阶群胚。而逻辑构造(就像积$A \times B$)则被视作这些空间之间的同伦不变量的构造。
这样一来, 我们得以直接操纵空间, 而无需首先发展点集拓扑(或任何它的组合替代理论, 例如单纯集论)
为了简要地介绍这种视角, 我们先考虑类型论的基础概念, 或确切地说:
\emph{项}$a$具有\emph{类型}$A$, 写作
\[
a : A
\]
传统上, 这样一个表达式被认为类似于
\begin{center}
``$a$是集合$A$的元素''
\end{center}
然而, 在同伦类型论中, 我们认为其类似于
\begin{center}
``$a$是空间$A$中一点''
\end{center}

\index{continuity of functions in type theory@``continuity'' of functions in type theory}%

类似地, 类型论中每个函数$f: A\to B$都被视作从空间$A$到空间$B$的一个连续映射。
需要强调的是, 这些``空间''都是在纯粹同伦论意义上对待的, 而不具有拓扑意义。
例如, 不存在类型的``开子集''的概念或是类型中元素列的``收敛性''。
我们只有``同伦的''概念, 例如点之间的道路与道路之间的同伦, 这些概念在同伦论的其它模型下也依然有意义(例如单纯集)
因此, 把类型视作\emph{$\infty$-群胚}\index{.infinity-groupoid@$\infty$-groupoid}是更加准确的说法。这是同伦论中``不变量对象''的名称, 可以用拓扑空间, 
\index{topological!space}%
单纯集, 或其他任何同伦论的模型表示。
然而, 偶尔使用拓扑学中``空间''和``道路''这样的名称是很方便的, 只要我们记得其他的拓扑概念是不可用的就好。

(为这些对象使用\emph{同伦类型}这个短语也是很诱人的, 
\index{homotopy!type}%
这暗示了一对对偶的解释: ``在同伦意义上被看待的(类型论的)类型'' 和 ``从同伦论的观点上看待空间''。
后者与经典意义上的``同伦类型''有些许不同——空间同伦等价意义下的\emph{等价类}, 尽管其确实保持了像``这两个空间具有相同的同伦类型''这样的短语的含义)

将类型视作结构化对象而非集合的想法有着悠长的历史, 众所周知, 它可以阐明类型论中许多神秘的方面。
例如, 将类型解释为层能够帮助解释类型论的直觉主义本质, 而将其解释为偏等价关系或``定义域''则有助于解释它的可计算性。
这也暗示着我们可以使用类型论的推理来研究结构化对象, 这指向了范畴论逻辑这一丰富的领域。
对其的同伦论解释也遵循了同样的模式: 它阐明了类型论中\emph{恒等}(或相等性)的本质, 允许我们在同伦论的研究中使用类型论式的推理。

同伦论解释的核心观点为: 对具有同样类型$A$的两个对象$a, b:A$,  逻辑上的恒等的概念$a=b$可以被理解为在空间$A$中存在一条从点$a$到$b$的d$p : a \leadsto b$。
这也意味着如果两个函数$f, g: A \to B$是同伦的, 它们就是恒等的。因为同伦只不过是$B$上的一个(连续)道路族$p_x: f(x) \leadsto g(x)$, 其中$x:A$。

在类型论中, 对每个类型$A$, 都有一个(在先前些许神秘的)类型$\idtypevar{A}$, 它是$A$上两个物体的等同证明的类型。
在同伦类型论中, 这只不过是从单位区间到空间$A$的全部连续映射$I \to A$所构成的\emph{道路空间}$A^I$
\index{unit!interval}%
\index{interval!topological unit}%
\index{path!topological}%
\index{topological!path}%
在这个意义上, 项$p : \idtype[A]{a}{b}$呈现了$A$上的一条道路$p : a \leadsto b$。

同伦类型论的思想在2006年左右, 在Awodey与Warren的合著~\cite{AW}, 和Voevodsky的著作~\cite{VV}中分别被提出,  
但其灵感源于Hofmann与Streicher的早期群胚解释~\cite{hs:gpd-typethy}。
实际上, 高阶范畴论(特别是弱$\infty$-群胚理论)与同伦论之间的亲密联系是众所周知的事。这一观点最先由Grothendieck提出, 现在同时被两个领域的数学家所关注。
Awodey--Warren和Voevodsky原先所提出的语义模型使用了同伦论中所熟知的概念与技巧。这些概念与技巧现在也在高阶范畴论中被使用, 
像是Quillen模型范畴和Kan\index{Kan complex}单纯集\index{simplicial!sets}。
\index{Quillen model category}%
\index{model category}%

实际上, Voevodsky构造了类型论的Kan单纯集解释, 并意识到这种解释满足一个更深刻的性质, 他称其为\emph{泛等}。
这一性质先前并未在类型论中被考虑过(尽管Church的命题外延性原理被发现是该性质的一个特例, 而Hofmann和Streicher之前考虑过另一个称为``宇宙外延性''的特例)。
将泛等以一种新公理的方式添加到类型论中有着深远的结果, 其中的许多是很自然, 简单且鼓舞人心的。
泛等公理也增强了同伦视角下的类型论, 因为它在单纯模型和其他相关模型中成立, 而在将类型视作集合的观点下不成立。

\subsection*{泛等基础}

简略地说, 泛等公理的基本思想可以解释如下:
在类型论中, 人们可以有一个元素本身为类型的一种类型; 这样一种类型叫做\emph{宇宙}并且通常记作$\UU$。
类型为$\UU$的对象的类型通常又称为\emph{小}类型。
\index{type!small}%
\index{small!type}%
像其他类型一样, $\UU$有着恒等类型$\idtypevar{\UU}$, 其表达了小类型间的恒等关系$A=B$。
将类型视作空间, $\UU$是一个点为空间的空间; 为了理解其恒等类型, 我们必须知道, $\UU$上空间$A$与$B$之间的道路$p: A \leadsto B$是什么?
泛等公理告诉我们, 这样的一条道路对应于同伦等价$\eqv A B$,  (大体上)就像前面所解释的那样。
更精确地说, 给定任意(小)类型$A$与$B$, 除了$A$与$B$之间的等同证明的原始类型$\idtype[\UU]AB$,还有定义类型$\texteqv AB$,其对象是从$A$到$B$的等价。
因为任何物体上的恒等映射都是一个等价, 因此有一个典范映射
\[\idtype[\UU]AB\to\texteqv AB.\]
泛等公理声称这样一个映射其本身也是一个等价。
尽管有着过度简化的风险, 我们可以将其简洁地描述如下

\begin{description}\index{univalence axiom}%
\item[Univalence Axiom:]  $\eqvspaced{(A = B)}{(\eqv A B)}$.
\end{description}
%
换言之, 恒等与等价等价。
特别地, 人们会说``等价的类型恒等''。
然而, 这样一个短语有些误导人, 因为这种说法听起来像是把等价的概念\emph{坍缩}至与恒等的概念重合, 像一种的``骨骼性''条件。而实际上, 泛等公理是把恒等的概念\emph{扩张}至与(未变的)等价的概念重合。
从同伦论的观点上看, 泛等蕴含着具有相同同伦类型的空间被宇宙$\UU$上的一条道路所连接, 这与(小)空间的分类空间的直观理解相符。
从逻辑性的观点上看, 这却是一个爆炸性的新思想: 它声称同构的事物可以是恒等的! 数学家在实践中总是理所当然的等同同构的事物, 但他们通常通过``滥用概念''\index{abuse!of notation}, 
或是通过其它非形式化的方法来这么做, 认为这些对象``实际上''并不恒等。然而在这个新基础的框架中, 这样的结构可以形式化为逻辑学上的恒等, 即每个涉及了其中一者的性质或构造可以应用于另一者。
实际上, 这样的等同证明现在可以被显式地作出, 而那些性质和构造可以沿着该证明被系统性地传递给另外一者。不仅如此, 等同证明的不同形式使得它们自身形成了一种可以(且应当!)\ 考虑的结构。

总之, 对宇宙$\UU$中的两点$A$与$B$(即小类型), 泛等公理将以下三个概念等同:
\begin{itemize}
\item (逻辑学) $A$与$B$间的等同证明$p:A=B$
\item (拓扑学) $\UU$上从$A$到$B$的一条道路$p:A \leadsto B$
\item (同伦论) $A$与$B$间的一个等价$p:\eqv A B$
\end{itemize}

\subsection*{高阶归纳类型}\index{type!higher inductive}%

One of the classical advantages of type theory is its simple and effective techniques for working with inductively defined structures.
The simplest nontrivial inductively defined structure is the natural numbers, which is inductively generated by zero and the successor function.
From this statement one can algorithmically\index{algorithm} extract the principle of mathematical induction, which characterizes the natural numbers.
More general inductive definitions encompass lists and well-founded trees of all sorts, each of which is characterized by a corresponding ``induction principle''.
This includes most data structures used in certain programming languages; hence the usefulness of type theory in formal reasoning about the latter.
If conceived in a very general sense, inductive definitions also include examples such as a disjoint union $A+B$, which may be regarded as ``inductively'' generated by the two injections $A\to A+B$ and $B\to A+B$.
The ``induction principle'' in this case is ``proof by case analysis'', which characterizes the disjoint union.

In homotopy theory, it is natural to consider also ``inductively defined spaces'' which are generated not merely by a collection of \emph{points}, but also by collections of \emph{paths} and higher paths.
Classically, such spaces are called \emph{CW complexes}.
\index{CW complex}%
For instance, the circle $S^1$ is generated by a single point and a single path from that point to itself.
Similarly, the 2-sphere $S^2$ is generated by a single point $b$ and a single two-dimensional path from the constant path at $b$ to itself, while the torus $T^2$ is generated by a single point, two paths $p$ and $q$ from that point to itself, and a two-dimensional path from $p\ct q$ to $q\ct p$.

By using the identification of paths with identities in Homotopy Type Theory, these sort of ``inductively defined spaces'' can be characterized in type theory by ``induction principles'', entirely analogously to classical examples such as the natural numbers and the disjoint union.
The resulting \emph{higher inductive types}
\index{type!higher inductive}%
give a direct ``logical'' way to reason about familiar spaces such as spheres, which (in combination with univalence) can be used to perform familiar arguments from homotopy theory, such as calculating homotopy groups of spheres, in a purely formal way.
The resulting proofs are a marriage of classical homotopy-theoretic ideas with classical type-theoretic ones, yielding new insight into both disciplines.

Moreover, this is only the tip of the iceberg: many abstract constructions from homotopy theory, such as homotopy colimits, suspensions, Postnikov towers, localization, completion, and spectrification, can also be expressed as higher inductive types.
Many of these are classically constructed using Quillen's ``small object argument'', which can be regarded as a finite way of algorithmically describing an infinite CW complex presentation\index{presentation!of a space as a CW complex} of a space, just as ``zero and successor'' is a finite algorithmic\index{algorithm} description of the infinite set of natural numbers.
Spaces produced by the small object argument are infamously complicated and difficult to understand; the type-theoretic approach is potentially much simpler, bypassing the need for any explicit construction by giving direct access to the appropriate ``induction principle''.
Thus, the combination of univalence and higher inductive types suggests the possibility of a revolution, of sorts, in the practice of homotopy theory.


\subsection*{泛等基础中的集合}

\index{set|(}%

We have claimed that univalent foundations can eventually serve as a foundation for ``all'' of mathematics, but so far we have discussed 
only homotopy theory.  Of course, there are many specific examples of the use of type theory without the new Homotopy Type Theory features to formalize mathematics,
\index{mathematics!formalized}%
\index{theorem!Feit--Thompson}%
\index{theorem!odd-order}%
\index{Feit--Thompson theorem}%
\index{odd-order theorem}%
such as the recent formalization of the Feit--Thompson odd-order theorem in \Coq~\cite{gonthier}.

But the traditional view is that mathematics is founded on set theory, in the sense that all mathematical objects and constructions can be coded into a theory such as Zermelo--Fraenkel set theory (ZF).
\index{set theory!Zermelo--Fraenkel}%
\indexsee{Zermelo-Fraenkel set theory}{set theory}%
\indexsee{ZF}{set theory}%
\indexsee{ZFC}{set theory}%
However, it is well-established by now that for most mathematics outside of set theory proper, the intricate hierarchical membership structure of sets in ZF is really unnecessary: a more ``structural'' theory, such as Lawvere's\index{Lawvere} Elementary Theory of the Category of Sets~\cite{lawvere:etcs-long}, suffices.
\index{Elementary Theory of the Category of Sets}%

In univalent foundations, the basic objects are ``homotopy types'' rather than sets, but we can \emph{define} a class of types which behave like sets.
Homotopically, these can be thought of as spaces in which every connected component is contractible, i.e.\ those which are homotopy equivalent to a discrete space.
\index{discrete!space}%
It is a theorem  that the category of such ``sets'' satisfies Lawvere's\index{Lawvere} axioms (or related ones, depending on the details of the theory).
Thus, any sort of mathematics that can be represented in an ETCS-like theory (which, experience suggests, is essentially all of mathematics) can equally well be represented in univalent foundations.  

This supports the claim that univalent foundations is at least as good as existing foundations of mathematics.
A mathematician working in univalent foundations can build structures out of sets in a familiar way, with more general homotopy types waiting in the foundational background until there is need of them.
For this reason, most of the applications in this book have been chosen to be areas where univalent foundations has something \emph{new} to contribute that distinguishes it from existing foundational systems.

Unsurprisingly, homotopy theory and category theory are two of these, but perhaps less obvious is that univalent foundations has something new and interesting to offer even in subjects such as set theory and real analysis.
For instance, the univalence axiom allows us to identify isomorphic structures, while higher inductive types allow direct descriptions of objects by their universal properties.
Thus we can generally avoid resorting to arbitrarily chosen representatives or transfinite iterative constructions.
In fact, even the objects of study in ZF set theory can be characterized, inside the sets of univalent foundations, by such an inductive universal property.

\index{set|)}%


\subsection*{非形式化集合论}

\index{mathematics!formalized|(defstyle}%
\index{informal type theory|(defstyle}%
\index{type theory!informal|(defstyle}%
\index{type theory!formal|(}%
One difficulty often encountered by the classical mathematician when faced with learning about type theory is that it is usually presented as a fully or partially formalized deductive system.
This style, which is very useful for proof-theoretic investigations, is not particularly convenient for use in applied, informal reasoning.
Nor is it even familiar to most working mathematicians, even those who might be interested in foundations of mathematics.
One objective of the present work is to develop an informal style of doing mathematics in univalent foundations that is at once rigorous and precise, but is also closer to the language and style of presentation of everyday mathematics.

In present-day mathematics, one usually constructs and reasons about mathematical objects in a way that could in principle, one presumes, be formalized in a system of elementary set theory, such as ZFC --- at least given enough ingenuity and patience.
For the most part, one does not even need to be aware of this possibility, since it largely coincides with the condition that a proof be ``fully rigorous'' (in the sense that all mathematicians have come to understand intuitively through education and experience).
But one does need to learn to be careful about a few aspects of ``informal set theory'': the use of collections too large or inchoate to be sets; the axiom of choice and its equivalents; even (for undergraduates) the method of proof by contradiction; and so on.
Adopting a new foundational system such as Homotopy Type Theory as the \emph{implicit formal basis} of informal reasoning will require adjusting some of one's instincts and practices.
The present text is intended to serve as an example of this ``new kind of mathematics'', which is still informal, but could now in principle be formalized in Homotopy Type Theory, rather than ZFC, again given enough ingenuity and patience.

It is worth emphasizing that, in this new system, such formalization can have real practical benefits.
The formal system of type theory is suited to computer systems and has been implemented in existing proof assistants.
\index{proof!assistant}%
A proof assistant is a computer program which guides the user in construction of a fully formal proof, only allowing valid steps of reasoning.
It also provides some degree of automation, can search libraries for existing theorems, and can even extract numerical algorithms\index{algorithm} \index{extraction of algorithms} from the resulting (constructive) proofs.

We believe that this aspect of the univalent foundations program distinguishes it from other approaches to foundations, potentially providing a new practical utility for the working mathematician.
Indeed, proof assistants based on older type theories have already been used to formalize substantial mathematical proofs, such as the four-color theorem\index{theorem!four-color} \index{four-color theorem} and the Feit--Thompson theorem.
Computer implementations of univalent foundations are presently works in progress (like the theory itself).
\index{proof!assistant}%
However, even its currently available implementations (which are mostly small modifications to existing proof assistants such as \Coq and 
\Agda) have already demonstrated their worth, not only in the formalization of known proofs, but in the discovery of new ones.
Indeed, many of the proofs described in this book were actually \emph{first} done in a fully formalized form in a proof assistant, and are only now being ``unformalized'' for the first time --- a reversal of the usual relation between formal and informal mathematics.

One can imagine a not-too-distant future when it will be possible for mathematicians to verify the correctness of their own papers by working within the system of univalent foundations, formalized in a proof assistant, and that doing so will become as natural as typesetting their own papers in \TeX.
%(Whether this proves to be the publishers' dream or their nightmare remains to be seen.) 
In principle, this could be equally true for any other foundational system, but we believe it to be more practically attainable using univalent foundations, as witnessed by the present work and its formal counterpart.

\index{type theory!formal|)}%
\index{informal type theory|)}%
\index{type theory!informal|)}%
\index{mathematics!formalized|)}%

\subsection*{可构造性} 

\index{mathematics!constructive|(}%

One of the most striking differences between classical\index{mathematics!classical} foundations and type theory is the idea of \emph{proof relevance}, according to which mathematical statements, and even their proofs, become first-class mathematical objects.
In type theory, we represent mathematical statements by types, which can be regarded simultaneously as both mathematical constructions and mathematical assertions, a conception also known as \emph{propositions as types}.
\index{proposition!as types}%
Accordingly, we can regard a term $a : A$ as both an element of the type $A$ (or in Homotopy Type Theory, a point of the space $A$), and at the same time, a proof of the proposition $A$.
To take an example, suppose we have sets $A$ and $B$ (discrete spaces),
\index{discrete!space}%
and consider the statement ``$A$ is isomorphic to $B$''.
In type theory, this can be rendered as:
\begin{narrowmultline*}
  \mathsf{Iso}(A,B) \defeq \narrowbreak
  \sm{f : A\to B}{g : B\to A}\Big(\big(\tprd{x:A} g(f(x)) = x\big) \times \big(\tprd{y:B}\, f(g(y)) = y\big)\Big).
\end{narrowmultline*}
%
Reading the type constructors $\Sigma, \Pi, \times$  here  as ``there exists'', ``for all'', and ``and'' respectively yields the usual formulation of ``$A$ and $B$ are isomorphic''; on the other hand, reading them as sums and products yields the \emph{type of all isomorphisms} between $A$ and $B$!  To prove that $A$ and $B$ are isomorphic, one  constructs a proof $p : \mathsf{Iso}(A,B)$, which is therefore the same  as constructing an isomorphism between $A$ and $B$, i.e., exhibiting a pair of functions $f, g$ together with \emph{proofs} that their composites are the respective identity maps.  The latter proofs, in turn, are nothing but homotopies of the appropriate sorts.  In this way, \emph{proving a proposition is the same as constructing an element of some particular type.}
In particular, to prove a statement of the form ``$A$ and $B$'' is just to prove $A$ and to prove $B$, i.e., to give an element of the type $A\times B$.
And to prove that $A$ implies $B$ is just to find an element of $A\to B$, i.e.\ a function from $A$ to $B$ (determining a mapping of proofs of $A$ to proofs of $B$).

The logic of propositions-as-types is flexible and supports many variations, such as using only a subclass of types to represent propositions.
In Homotopy Type Theory, there are natural such subclasses arising from the fact that the system of all types, just like spaces in classical homotopy theory, is ``stratified'' according to the dimensions in which their higher homotopy structure exists or collapses.
In particular, Voevodsky has found a purely type-theoretic definition of \emph{homotopy $n$-types}, corresponding to spaces with no nontrivial homotopy information above dimension $n$.
(The $0$-types are the ``sets'' mentioned previously as satisfying Lawvere's axioms\index{Lawvere}.)
Moreover, with higher inductive types, we can universally ``truncate'' a type into an $n$-type; in classical homotopy theory this would be its $n^{\mathrm{th}}$ Postnikov\index{Postnikov tower} section.\index{n-type@$n$-type}
Particularly important for logic is the case of homotopy $(-1)$-types, which we call \emph{mere propositions}.
Classically, every $(-1)$-type is empty or contractible; we interpret these possibilities as the truth values ``false'' and ``true'' respectively.

Using all types as propositions yields a very ``constructive'' conception of logic; for more on this, see~\cite{kolmogorov,TroelstraI,TroelstraII}.
For instance, every proof that something exists carries with it enough information to actually find such an object; and every proof that ``$A$ or $B$'' holds is either a proof that $A$ holds or a proof that $B$ holds.
Thus, from every proof we can automatically extract an algorithm;\index{algorithm} \index{extraction of algorithms} this can be very useful in applications to computer programming.

On the other hand, however, this logic does diverge from the traditional understanding of existence proofs in mathematics.
In particular, it does not faithfully represent certain important classical principles of reasoning, such as the axiom of choice and the law of excluded middle.
For these we need to use the ``$(-1)$-truncated'' logic, in which only the homotopy $(-1)$-types represent propositions.

\index{axiom!of choice}%
More specifically, consider on one hand the \emph{axiom of choice}: ``if for every $x: A$ there exists a $y:B$ such that $R(x,y)$, there is a function $f : A\to B$ such that for all $x:A$ we have $R(x, f(x))$.''
The pure propositions-as-types notion of ``there exists'' is strong enough to make this statement simply provable --- yet it does not have all the consequences of the usual axiom of choice.
However, in $(-1)$-truncated logic, this statement is not automatically true, but is a strong assumption with the same sorts of consequences as its counterpart in classical\index{mathematics!classical} set theory.

\index{excluded middle}%
\index{univalence axiom}%
On the other hand, consider the \emph{law of excluded middle}: ``for all $A$, either $A$ or not $A$.''
Interpreting this in the pure propositions-as-types logic yields a statement that is inconsistent with the univalence axiom.
For since proving ``$A$'' means exhibiting an element of it, this assumption would give a uniform way of selecting an element from every nonempty type --- a sort of Hilbertian choice operator.
Univalence implies that the element of $A$ selected by such a choice operator must be invariant under all self-equivalences of $A$, since these are identified with self-identities and every operation must respect identity; but clearly some types have automorphisms with no fixed points, e.g.\ we can swap the elements of a two-element type.
\index{automorphism!fixed-point-free}%
However, the ``$(-1)$-truncated law of excluded middle'', though also not automatically true, may consistently be assumed with most of the same consequences as in classical mathematics.

In other words, while the pure propositions-as-types logic is ``constructive'' in the strong algorithmic sense mentioned above, the default $(-1)$-truncated logic is ``constructive'' in a different sense (namely, that of the logic formalized by Heyting under the name ``intuitionistic''); and to the latter we may freely add the axioms of choice and excluded middle to obtain a logic that may be called ``classical''.
Thus, Homotopy Type Theory is compatible with both constructive and classical conceptions of logic, and many more besides.
\index{logic!constructive vs classical}%
Indeed, the homotopical perspective reveals that classical and constructive logic can coexist, as endpoints of a spectrum of different systems, with an infinite number of possibilities in between (the homotopy $n$-types for $-1 < n < \infty$).
We may speak of ``\LEM{n}'' and ``\choice{n}'', with $\choice{\infty}$ being provable and \LEM{\infty} inconsistent with univalence, while $\choice{-1}$ and $\LEM{-1}$ are the versions familiar to classical mathematicians (hence in most cases it is appropriate to assume the subscript $(-1)$ when none is given).  Indeed, one can even have useful systems in which only \emph{certain} types satisfy such further ``classical'' principles, while types in general remain ``constructive''.\index{excluded middle}\index{axiom!of choice}%%

It is worth emphasizing that univalent foundations does not \emph{require} the use of constructive or intuitionistic logic.\index{logic!intuitionistic}\index{logic!constructive} %
Most of classical mathematics which depends on the law of excluded middle and the axiom of choice can be performed in univalent foundations, simply by assuming that these two principles hold (in their proper, $(-1)$-truncated, form).
However, type theory does encourage avoiding these principles when they are unnecessary, for several reasons.

First of all, every mathematician knows that a theorem is more powerful when proven using fewer assumptions, since it applies to more examples.
The situation with \choice{} and \LEM{} is no different:
type theory admits many interesting ``nonstandard'' models, such as in sheaf toposes,\index{topos} where classicality principles such as \choice{} and \LEM{} tend to fail.
Homotopy Type Theory admits similar models in higher toposes, such as are studied in~\cite{ToenVezzosi02,Rezk05,lurie:higher-topoi}.
Thus, if we avoid using these principles, the theorems we prove will be valid internally to all such models.

Secondly, one of the additional virtues of type theory is its computable character.
In addition to being a foundation for mathematics, type theory is a formal theory of computation, and can be treated as a powerful programming language.
\index{programming}%
From this perspective, the rules of the system cannot be chosen arbitrarily the way set-theoretic axioms can: there must be a harmony between them which allows all proofs to be ``executed'' as programs.
We do not yet fully understand the new principles introduced by Homotopy Type Theory, such as univalence and higher inductive types, from
this point of view, but the basic outlines are emerging; see, for example,~\cite{lh:canonicity}.
It has been known for a long time, however, that principles such as \choice{} and \LEM{} are fundamentally antithetical to computability, since they assert baldly that certain things exist without giving any way to compute them.
Thus, avoiding them is necessary to maintain the character of type theory as a theory of computation.

Fortunately, constructive reasoning is not as hard as it may seem.
In some cases, simply by rephrasing some definitions, a theorem can be made constructive and its proof more elegant.
Moreover, in univalent foundations this seems to happen more often.
For instance:
\begin{enumerate}
\item In set-theoretic foundations, at various points in homotopy theory and category theory one needs the axiom of choice to perform transfinite constructions.
  But with higher inductive types, we can encode these constructions directly and constructively.
  In particular, none of the ``synthetic'' homotopy theory in \cref{cha:homotopy} requires \LEM{} or \choice{}.
\item In set-theoretic foundations, the statement ``every fully faithful and essentially surjective functor is an equivalence of categories'' is equiv\-a\-lent to the axiom of choice.
  But with the univalence axiom, it is just \emph{true}; see \cref{cha:category-theory}.
\item In set theory, various circumlocutions are required to obtain notions of ``cardinal number'' and ``ordinal number'' which canonically represent isomorphism classes of sets and well-ordered sets, respectively --- possibly involving the axiom of choice or the axiom of foundation.
  But with univalence and higher inductive types, we can obtain such representatives directly by truncating the universe; see \cref{cha:set-math}.
\item In set-theoretic foundations, the definition of the real numbers as equivalence classes of Cauchy sequences requires either the law of excluded middle or the axiom of (countable) choice to be well-behaved.
  But with higher inductive types, we can give a version of this definition which is well-behaved and avoids any choice principles; see \cref{cha:real-numbers}.
\end{enumerate}
Of course, these simplifications could as well be taken as evidence that the new methods will not, ultimately, prove to be really constructive.  However, we emphasize again that the reader does not have to care, or worry, about constructivity in order to read this book.  The point is that in all of the above examples, the version of the theory we give has independent advantages, whether or not \LEM{} and \choice{} are assumed to be available.  Constructivity, if attained, will be an added bonus.\index{constructivity}%

Given this discussion of adding new principles such as univalence, higher inductive types, \choice{}, and \LEM{}, one may wonder whether the resulting system remains consistent.
(One of the original virtues of type theory, relative to set theory, was that it can be seen to be consistent by proof-theoretic means).
As with any foundational system, consistency\index{consistency} is a relative question: ``consistent with respect to what?''
The short answer is that all of the constructions and axioms considered in this book have a model in the category of Kan\index{Kan complex} complexes, due to Voevodsky~\cite{klv:ssetmodel} (see~\cite{ls:hits} for higher inductive types).
Thus, they are known to be consistent relative to ZFC (with as many inaccessible cardinals
\index{inaccessible cardinal}\index{consistency}%
as we need nested univalent universes).
Giving a more traditionally type-theoretic account of this consistency is work in progress (see,
e.g.,~\cite{lh:canonicity,coquand2012constructive}).

We summarize the different points of view of the type-theoretic operations in \cref{tab:pov}.

\begin{table}[htb]
  \centering
  \OPTsmalltable
 \begin{tabular}{lllll}
    \toprule
       Types && Logic & Sets & Homotopy\\ \addlinespace[2pt]
    \midrule
       $A$ && proposition & set & space\\ \addlinespace[2pt]
       $a:A$ && proof & element & point \\ \addlinespace[2pt]
       $B(x)$ && predicate & family of sets & fibration \\ \addlinespace[2pt]
       $b(x) : B(x)$ && conditional proof & family of elements & section\\ \addlinespace[2pt]
       $\emptyt, \unit$ && $\bot, \top$ & $\emptyset, \{ \emptyset \}$ & $\emptyset, *$\\ \addlinespace[2pt]
       $A + B$ && $A\vee B$ & disjoint union & coproduct\\ \addlinespace[2pt]
       $A\times B$ && $A\wedge B$ & set of pairs & product space\\ \addlinespace[2pt]
       $A\to B$ && $A\Rightarrow B$ & set of functions & function space\\ \addlinespace[2pt]
       $\sm{x:A}B(x)$ &&  $\exists_{x:A}B(x)$ & disjoint sum & total space\\ \addlinespace[2pt]
       $\prd{x:A}B(x)$ &&  $\forall_{x:A}B(x)$ & product & space of sections\\ \addlinespace[2pt]
       $\mathsf{Id}_{A}$ && equality $=$ & $\setof{\pairr{x,x} | x\in A}$ & path space $A^I$ \\ \addlinespace[2pt]
    \bottomrule
  \end{tabular}
  \caption{Comparing points of view on type-theoretic operations}\label{tab:pov}
\end{table}

\index{mathematics!constructive|)}%

\subsection*{开放问题} 

\index{open!problem|(}%

For those interested in contributing to this new branch of mathematics, it may be encouraging to know that there are many interesting open questions.

\index{univalence axiom!constructivity of}%
Perhaps the most pressing of them is the ``constructivity'' of the Univalence Axiom, posed by Voevodsky in \cite{Universe-poly}.
The basic system of type theory follows the structure of Gentzen's natural deduction. Logical connectives are defined by their introduction rules, and have elimination rules justified by computation rules. Following this pattern, and using Tait's computability method, originally designed to analyse G\"odel's Dialectica interpretation, one can show the property of \emph{normalization} for type theory. This in turn implies important properties such as decidability of type-checking (a crucial property since type-checking corresponds to proof-checking, and one can argue that we should be able to ``recognize a proof when we see one''), and the so-called ``canonicity\index{canonicity} property'' that any closed term of the type of natural numbers reduces to a numeral. This last property, and the uniform structure of introduction/elimination rules, are lost when one extends type theory with an axiom, such as the axiom of function extensionality, or the univalence axiom. Voevodsky has formulated a precise mathematical conjecture connected to this question of canonicity for type theory extended with the axiom of Univalence: given a closed term of the type of natural numbers, is it always possible to find a numeral and a proof that this term is equal to this numeral, where this proof of equality may itself use the univalence axiom? More generally, an important issue is whether it is possible to provide a constructive justification of the univalence axiom.
What about if one adds other homotopically motivated constructions, like higher inductive types?
These questions remain open at the present time, although methods are currently being developed to try to find answers.

Another basic issue is the difficulty of working with types, such as the natural numbers, that are essentially sets (i.e., discrete spaces),
\index{discrete!space}%
containing only trivial paths.
At present, Homotopy Type Theory can really only characterize spaces up to homotopy equivalence, which means that these ``discrete spaces'' may only be \emph{homotopy equivalent} to discrete spaces.
Type-theoretically, this means there are many paths that are equal to reflexivity, but not \emph{judgmentally} equal to it (see \cref{sec:types-vs-sets} for the meaning of ``judgmentally'').
While this homotopy-invariance has advantages, these ``meaningless'' identity terms do introduce needless complications into arguments and constructions, so it would be convenient to have a systematic way of eliminating or collapsing them.
% In some cases, the proliferation of such superfluous identity terms makes it very difficult or impossible to formulate what should be a straightforward concept, such as the definition of a (semi-)simplicial type.

A more specialized, but no less important, problem is the relation between Homotopy Type Theory and the research on \emph{higher toposes}%
\index{.infinity1-topos@$(\infty,1)$-topos}
currently happening at the intersection of higher category theory and homotopy theory.
There is a growing conviction among those familiar with both subjects that they are intimately connected.
For instance, the notion of a univalent universe should coincide with that of an object classifier, while higher inductive types should be an ``elementary'' reflection of local presentability.
More generally, Homotopy Type Theory should be the ``internal language'' of $(\infty,1)$-toposes, just as intuitionistic higher-order logic is the internal language of ordinary 1-toposes.
Despite this general consensus, however, details remain to be worked out --- in particular, questions of coherence and strictness remain to be addressed  --- and doing so will undoubtedly lead to further insights into both concepts.

\index{mathematics!formalized}%
But by far the largest field of work to be done is in the ongoing formalization of everyday mathematics in this new system.
Recent successes in formalizing some facts from basic homotopy theory and category theory have been encouraging; some of these are described in \cref{cha:homotopy,cha:category-theory}.
Obviously, however, much work remains to be done.

\index{open!problem|)}%

The Homotopy Type Theory community maintains a web site and group blog at \url{http://homotopytypetheory.org}, as well as a discussion email list.
Newcomers are always welcome!


\subsection*{如何阅读本书}

This book is divided into two parts.
\cref{part:foundations}, ``Foundations'', develops the fundamental concepts of Homotopy Type Theory.
This is the mathematical foundation on which the development of specific subjects is built, and which is required for the understanding of the univalent foundations approach. To a programmer, this is ``library code''.
Since univalent foundations is a new and different kind of mathematics, its basic notions take some getting used to; thus \cref{part:foundations} is fairly extensive.

\cref{part:mathematics}, ``Mathematics'', consists of four chapters that build on the basic notions of \cref{part:foundations} to exhibit some of the new things we can do with univalent foundations in four different areas of mathematics: homotopy theory (\cref{cha:homotopy}), category theory (\cref{cha:category-theory}), set theory (\cref{cha:set-math}), and real analysis (\cref{cha:real-numbers}).
The chapters in \cref{part:mathematics} are more or less independent of each other, although occasionally one will use a lemma proven in another.

A reader who wants to seriously understand univalent foundations, and be able to work in it, will eventually have to read and understand most of \cref{part:foundations}.
However, a reader who just wants to get a taste of univalent foundations and what it can do may understandably balk at having to work through over 200 pages before getting to the ``meat'' in \cref{part:mathematics}.
Fortunately, not all of \cref{part:foundations} is necessary in order to read the chapters in \cref{part:mathematics}.
Each chapter in \cref{part:mathematics} begins with a brief overview of its subject, what univalent foundations has to contribute to it, and the necessary background from \cref{part:foundations}, so the courageous reader can turn immediately to the appropriate chapter for their favorite subject.
For those who want to understand one or more chapters in \cref{part:mathematics} more deeply than this, but are not ready to read all of \cref{part:foundations}, we provide here a brief summary of \cref{part:foundations}, with remarks about which parts are necessary for which chapters in \cref{part:mathematics}.

\cref{cha:typetheory} is about the basic notions of type theory, prior to any homotopical interpretation.
A reader who is familiar with Martin-L\"of type theory can quickly skim it to pick up the particulars of the theory we are using.
However, readers without experience in type theory will need to read \cref{cha:typetheory}, as there are many subtle differences between type theory and other foundations such as set theory.

\cref{cha:basics} introduces the homotopical viewpoint on type theory, along with the basic notions supporting this view, and describes the homotopical behavior of each component of the type theory from \cref{cha:typetheory}.
It also introduces the \emph{univalence axiom} (\cref{sec:compute-universe}) --- the first of the two basic innovations of Homotopy Type Theory.
Thus, it is quite basic and we encourage everyone to read it, especially \crefrange{sec:equality}{sec:basics-equivalences}.

\cref{cha:logic} describes how we represent logic in Homotopy Type Theory, and its connection to classical logic as well as to constructive and intuitionistic logic.
Here we define the law of excluded middle, the axiom of choice, and the axiom of propositional resizing (although, for the most part, we do not need to assume any of these in the rest of the book), as well as the \emph{propositional truncation} which is essential for representing traditional logic.
This chapter is essential background for \cref{cha:set-math,cha:real-numbers}, less important for \cref{cha:category-theory}, and not so necessary for \cref{cha:homotopy}.

\cref{cha:equivalences,cha:induction} study two special topics in detail: equivalences (and related notions) and generalized inductive definitions.
While these are important subjects in their own rights and provide a deeper understanding of Homotopy Type Theory, for the most part they are not necessary for \cref{part:mathematics}.
Only a few lemmas from \cref{cha:equivalences} are used here and there, while the general discussions in \cref{sec:bool-nat,sec:strictly-positive,sec:generalizations} are helpful for providing the intuition required for \cref{cha:hits}.
The generalized sorts of inductive definition discussed in \cref{sec:generalizations} are also used in a few places in \cref{cha:set-math,cha:real-numbers}.

\cref{cha:hits} introduces the second basic innovation of Homotopy Type Theory --- \emph{higher inductive types} --- with many examples.
Higher inductive types are the primary object of study in \cref{cha:homotopy}, and some particular ones play important roles in \cref{cha:set-math,cha:real-numbers}.
They are not so necessary for \cref{cha:category-theory}, although one example is used in \cref{sec:rezk}.

Finally, \cref{cha:hlevels} discusses homotopy $n$-types and related notions such as $n$-connected types.
These notions are important for \cref{cha:homotopy}, but not so important in the rest of \cref{part:mathematics}, although the case $n=-1$ of some of the lemmas are used in \cref{sec:piw-pretopos}.

This completes \cref{part:foundations}.
As mentioned above, \cref{part:mathematics} consists of four largely unrelated chapters, each describing what univalent foundations has to offer to a particular subject.

Of the chapters in \cref{part:mathematics}, \cref{cha:homotopy} (Homotopy theory) is perhaps the most radical.
Univalent foundations has a very different ``synthetic'' approach to homotopy theory in which homotopy types are the basic objects (namely, the types) rather than being constructed using topological spaces or some other set-theoretic model.
This enables new styles of proof for classical theorems in algebraic topology, of which we present a sampling, from $\pi_1(\Sn^1)=\Z$ to the Freudenthal suspension theorem.

In \cref{cha:category-theory} (Category theory), we develop some basic (1-)category theory, adhering to the principle of the univalence axiom that \emph{equality is isomorphism}.
This has the pleasant effect of ensuring that all definitions and constructions are automatically invariant under equivalence of categories: indeed, equivalent categories are equal just as equivalent types are equal.
(It also has connections to higher category theory and higher topos theory.)

\cref{cha:set-math} (Set theory) studies sets in univalent foundations.
The category of sets has its usual properties, hence provides a foundation for any mathematics that doesn't need homotopical or higher-categorical structures.
We also observe that univalence makes cardinal and ordinal numbers a bit more pleasant, and that higher inductive types yield a cumulative hierarchy satisfying the usual axioms of Zermelo--Fraenkel set theory.

In \cref{cha:real-numbers} (Real numbers), we summarize the construction of Dedekind real numbers, and then observe that higher inductive types allow a definition of Cauchy real numbers that avoids some associated problems in constructive mathematics.
Then we sketch a similar approach to Conway's surreal numbers.

Each chapter in this book ends with a Notes section, which collects historical comments, references to the literature, and attributions of results, to the extent possible.
We have also included Exercises at the end of each chapter, to assist the reader in gaining familiarity with doing mathematics in univalent foundations.

Finally, recall that this book was written as a massively collaborative effort by a large number of people.
We have done our best to achieve consistency in terminology and notation, and to put the mathematics in a linear sequence that flows logically, but it is very likely that some imperfections remain.
We ask the reader's forgiveness for any such infelicities, and welcome suggestions for improvement of the next edition.


% Local Variables:
% TeX-master: "hott-online"
% End:
