\chapter*{前言}
\label{cha:preface}

% Uncomment this if you think Preface should appear in TOC
%\addcontentsline{toc}{chapter}{Preface}
\subsection*{普林斯顿高等研究院“泛等基础”主题年}

普林斯顿高等研究院“泛等基础”主题年于2012-13年在普林斯顿高等研究所数学院举行。由Steve Awodey, Thierry Coquand与Vladimir Voevodsky组织。

以下为官方参与者:

\begin{multicols}{\OPTprefacecols}{
\begin{itemize}
\item[] Peter Aczel
\item[] Benedikt Ahrens
\item[] Thorsten Altenkirch
\item[] Steve Awodey
\item[] Bruno Barras
\item[] Andrej Bauer
\item[] Yves Bertot
\item[] Marc Bezem
\item[] Thierry Coquand
\item[] Eric Finster
\item[] Daniel Grayson
\item[] Hugo Herbelin
\item[] Andr\'e Joyal
\item[] Dan Licata
\item[] Peter Lumsdaine
\item[] Assia Mahboubi
\item[] Per Martin-L\"of
\item[] Sergey Melikhov
\item[] Alvaro Pelayo
\item[] Andrew Polonsky
\item[] Michael Shulman
\item[] Matthieu Sozeau
\item[] Bas Spitters
\item[] Benno van den Berg
\item[] Vladimir Voevodsky
\item[] Michael Warren
\item[] Noam Zeilberger
\end{itemize}
}
\end{multicols}

\noindent 此外,还有以下学生,他们的参与同样宝贵

\begin{multicols}{\OPTprefacecols}{
\begin{itemize}
\item[] Carlo Angiuli
\item[] Anthony Bordg
\item[] Guillaume Brunerie
\item[] Chris Kapulkin
\item[] Egbert Rijke
\item[] Kristina Sojakova
\end{itemize}
}
\end{multicols}

\noindent 此外,下列短期与长期来访者(包括学生来访者)对主题年的贡献亦不可或缺。

\begin{multicols}{\OPTprefacecols}{
\begin{itemize}
\item[] Jeremy Avigad
\item[] Cyril Cohen
\item[] Robert Constable
\item[] Pierre-Louis Curien
\item[] Peter Dybjer
\item[] Mart{\'\i}n Escard{\'o}
\item[] Kuen-Bang Hou
\item[] Nicola Gambino
\item[] Richard Garner
\item[] Georges Gonthier
\item[] Thomas Hales
\item[] Robert Harper
\item[] Martin Hofmann
\item[] Pieter Hofstra
\item[] Joachim Kock
\item[] Nicolai Kraus
\item[] Nuo Li
\item[] Zhaohui Luo
\item[] Michael Nahas
\item[] Erik Palmgren
\item[] Emily Riehl
\item[] Dana Scott
\item[] Philip Scott
\item[] Sergei Soloviev
\end{itemize}
}
\end{multicols}

\subsection*{关于此书}

写书并非我们的本意。现在的工作源于我们的一次集体尝试。我们试图构造一种能够被人类所理解的新型“非形式化类型论”,作为能被机器检查的形式化类型论的补充。
泛等基础与一种能被计算机证明助手所实现的数学基础关系密切。尽管形式化并非本书的一部分,此处呈现的许多材料都是先在一个证明助手中完全形式化后
再“非形式化”为你所看见的表述——这与形式化数学的通常做法明显相反。

上述每一个人都以思想、言辞或行动的方式对主题年与本书作出了贡献。
贯穿全年的合作精神着实非同凡响。

\mentalpause

特别感谢普林斯顿高等研究所。没有其协助,这本书永远不可能出现。
事实证明,这里是创造新数学的理想环境:激励人心,氛围融洽且支持性强
希望这样一种氛围的踪迹能萦绕在这本书的书页,与这一个新领域的进一步的发展中。

\bigskip

\begin{flushright}
泛等基础纲领\\
普林斯顿高等研究院\\
普林斯顿, 2013年四月
\end{flushright}

%%%%%%%%%%%%% end of scope of local macros
% Local Variables:
% TeX-master: "hott-online"
% End:
