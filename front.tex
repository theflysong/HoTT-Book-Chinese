%%%%%%%%%%%%%%%%%%%% Cover page %%%%%%%%%%%%%%%%%%%%

\newgeometry{noheadfoot,bindingoffset=-5pt,top=0pt,bottom=0pt,inner=0pt,outer=0pt}%
\ifOPTcover
\setcounter{page}{-1} % Otherwise we end up having two pages numbered 1
\newlength{\coverheight}
\setlength{\coverheight}{\OPTcoverheight}
\newlength{\coverwidth}
\setlength{\coverwidth}{\OPTcoverwidth}
\newcommand{\frontpage}{
\begin{minipage}[b][\coverheight][c]{\coverwidth}
\hbox{}\hfill
\begin{minipage}[b][\coverheight][t]{0.83\coverwidth}
\color{covertext}
\vspace{\OPTtopskip}
{\fontsize{\OPTcovertitlefont}{\OPTcovertitlefont}\fontseries{b}\selectfont%
  \hfill 同伦\par \hfill 类型论}\par
\vspace*{\OPTcovertitleskip}
{\fontsize{\OPTcoversubtitlefont}{\OPTcoversubtitlefont}\fontshape{it}\selectfont
\hfill 数学泛等基础}

\vfill

{\fontsize{\OPTcoverauthorfont}{\OPTcoverauthorfont}\fontseries{b}\fontshape{sc}\selectfont

\hfill 泛等基础纲领
\par
\vspace*{\OPTcoverauthorskip}
\hfill 普林斯顿高等研究院 \par

\vspace*{\OPTbotskip}
}

\end{minipage}\hfill\hbox{}
\end{minipage}
}

\ThisLLCornerWallPaper{1.1}{\OPTfrontimage}
\pagecolor{covercolor}
\frontpage
\newpage
% Reset page counter, cover page does not count
\ifpdf
\nopagecolor
\else
\pagecolor{white}
\fi
\cleartooddpage
\else
\fi

%%%%%%%%%%%%%%%%%%%% Bastard page %%%%%%%%%%%%%%%%%%%%
\ifOPTbastard
\cleartooddpage
\hbox{}
\vspace{0.2\textwidth}
{\centering
\makebox[\OPTbastardwidth][s]{
\fontsize{\OPTbastardtitlefont}{\OPTbastardtitlefont}\fontshape{n}\selectfont%
\textbf{同伦类型论}}\par
\vspace*{\OPTbastardtitleskip}
\makebox[\OPTbastardwidth][s]{
\fontsize{\OPTbastardsubtitlefont}{\OPTbastardsubtitlefont}\fontshape{n}\selectfont%
\textit{数学泛等基础}}\par
}
\else
\fi

%%%%%%%%%%%%%%%%%%%% Title page %%%%%%%%%%%%%%%%%%%%
\cleartooddpage
\hbox{}\vfill
{\centering
\makebox[\OPTtitlewidth][s]{\fontsize{\OPTtitletitlefont}{\OPTtitletitlefont}\fontseries{b}\selectfont%
同伦类型论}\par
\vspace*{\OPTtitletitleskip}
\makebox[\OPTtitlewidth][s]{\fontsize{\OPTtitlesubtitlefont}{\OPTtitlesubtitlefont}\fontshape{it}\selectfont%
数学泛等基础}\par
\vspace*{\OPTtitleskip}
{\fontsize{\OPTtitleauthorfont}{\OPTtitleauthorfont}\fontshape{n}\selectfont%
泛等基础纲领\par
\vspace*{\OPTtitleauthorskip}
普林斯顿高等研究院\par
}
\vspace*{\OPTtitleskip}
\vspace*{\OPTtitleskip}
\includegraphics[width=\OPTtitlewidth]{\OPThalftorus}\par
}

\vfill
\hbox{}

\clearpage
%%% Restore page style
\restoregeometry

%%%%%%%%%%%%%%%%%%%% Copyright page %%%%%%%%%%%%%%%%%%%%
\hbox{}
\vfill
\input{version.tex}
{\small
\noindent
\emph{``同伦类型论:数学泛等基础''}\\
\copyright\ 2013 泛等基础纲领

\medskip
\noindent
版本: \texttt{\OPTversion}

\medskip
\noindent
MSC 2010 分类号:
\texttt{03-02},
\texttt{55-02},
\texttt{03B15}

\bigskip
\footnotesize

\noindent
本作品采用的许可协议为
\textbf{\emph{Creative Commons Attribution-ShareAlike 3.0 Unported License.}}
%
协议副本见
\url{http://creativecommons.org/licenses/by-sa/3.0/}.

\bigskip

\noindent
本书可在 \url{http://homotopytypetheory.org/book/}. 中免费获得

\bigskip

\noindent
\emph{\textbf{\small 致谢}}

\medskip

\noindent
除了普林斯顿高等研究所的慷慨支持,本书的部分撰稿人还得到了下列机构和基金的部分或全部资助
%
\begin{itemize}
\item 普林斯顿高等研究所成员协会: 向普林斯顿高等研究院提供资助 % Dan Grayson
% SLOVENIA
\item 斯洛文尼亚研究署:  % Andrej's Slovenian agency
\href{http://www.sicris.si/search/prg.aspx?id=6120}{P1--0294},
\href{http://www.sicris.si/search/prj.aspx?id=7109}{N1--0011}.

\item 美国空军科学研究办公室:
  FA9550-11-1-0143, and % Steve's ASFOR
  FA9550-12-1-0370.  % Bob's ASFOR
  {
    \setlength{\parskip}{0pt}
    \begin{quote}
      \noindent\scriptsize
      本材料部分基于上述奖项下由美国空军科学研究办公室资助的工作。
      本出版物中表达的任何观点、发现、结论或建议均属于作者个人意见,并不一定反映美国空军科学研究办公室的立场。
    \end{quote}
  }

\item 英国工程与物理科学研究委员会: % Thorsten and students
   \href{http://gow.epsrc.ac.uk/NGBOViewGrant.aspx?GrantRef=EP/G034109/1}{EP/G034109/1}, % Reusability and dependent types
   \href{http://gow.epsrc.ac.uk/NGBOViewGrant.aspx?GrantRef=EP/G03298X/1}{EP/G03298X/1}. % Theory and Application of Induction Recursion

\item 由欧盟第七框架计划资助(资助协议号 243847) (%
\href{http://wiki.portal.chalmers.se/cse/pmwiki.php/ForMath/ForMath/}{ForMath}). %% several Europeans, via Bas

\item 美国国家科学基金会:
  \href{http://www.nsf.gov/awardsearch/showAward.do?AwardNumber=1001191}{DMS-1001191}, %% Steve's NSF, including Chris and Kristina
  \href{http://www.nsf.gov/awardsearch/showAward.do?AwardNumber=1100938}{DMS-1100938}, %% Vladimir's NSF
  \href{http://www.nsf.gov/awardsearch/showAward.do?AwardNumber=1116703}{CCF-1116703}, %% Foundations and Applications of Higher-Dimensional Directed Type Theory
  与
  \href{http://www.nsf.gov/awardsearch/showAward.do?AwardNumber=1128155}{DMS-1128155}. %% IAS support for Mike Shulman, copied from our %% paper by Dan Licata
  {
    \setlength{\itemsep}{0pt}
    \begin{quote}
      \noindent\scriptsize
      本研究成果部分基于美国国家科学基金会在上述奖项下的资助。作者在本材料中阐述的任何意见、发现、结论或建议均不代表美国国家科学基金会的观点。
    \end{quote}
  }
\item 西蒙尼基金:向普林斯顿高等研究院提供资助          %% Dan Grayson
  \end{itemize}


}
\cleartooddpage

%%% Local Variables:
%%% mode: latex
%%% TeX-master: "hott-online"
%%% End:
